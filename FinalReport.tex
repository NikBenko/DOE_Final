
\documentclass[11pt, twocolumn]{article}
\usepackage{amsmath}
\usepackage{amsfonts}
\usepackage{mathrsfs}
\usepackage{lscape}
\usepackage{listings}
\usepackage{graphicx} % Allows for importing of figures
\usepackage{color} % Allows for fonts to be colored
\usepackage{comment} % Allows for comments to be made
\usepackage{accents} % Allows for accents to be made above and below text
%\usepackage{undertilde} % Allows for under tildes to take place for vectors and tensors
\usepackage[table]{xcolor}
\usepackage{array,ragged2e}
\usepackage{hyperref}
\usepackage{framed} % Allows boxes to encase equations and such
\usepackage{subcaption} % Allows for figures to be side-by-side
\usepackage{float} % Allows for images to not float in the document
\usepackage{booktabs}
%\usepackage[margin=0.75in]{geometry}
\usepackage[final]{pdfpages}
\usepackage{enumitem}
\usepackage[section]{placeins}
\usepackage{cite}
%%%%%%%%%%%%%%%%%%%%%%%%%  Function used to generate vectors and tensors %%%%%%%%%
\usepackage{stackengine}
\stackMath
\newcommand\tensor[2][1]{%
	\def\useanchorwidth{T}%
	\ifnum#1>1%
	\stackunder[0pt]{\tensor[\numexpr#1-1\relax]{#2}}{\scriptscriptstyle \sim}%
	\else%
	\stackunder[1pt]{#2}{\scriptscriptstyle \sim}%
	\fi%
}
%%%%%%%%%%%%%%%%%%%

\definecolor{mygrey}{rgb}{0.97,0.98,0.99}
\definecolor{codeblue}{rgb}{.2,0,1}
\definecolor{codered}{rgb}{1,0,0}
\definecolor{codegreen}{rgb}{0.3,0.33,0.12}
\definecolor{codegray}{rgb}{0.5,0.5,0.5}
\definecolor{codepurple}{rgb}{0.55,0.0,0.55}
\definecolor{codecyan}{rgb}{0.0,.4,.4}

\lstdefinestyle{mystyle}{
	backgroundcolor=\color{mygrey},   
	commentstyle=\color{codegreen},
	keywordstyle=\color{codeblue},
	stringstyle=\color{codepurple},
	numberstyle=\tiny\color{codegray},
	basicstyle=\footnotesize,
	breakatwhitespace=false,         
	breaklines=true,                 
	captionpos=b,                    
	keepspaces=true, 
	numbers=left,                    
	numbersep=5pt,                  
	showspaces=false,                
	showstringspaces=false,
	showtabs=false,                  
	tabsize=2
}
\lstset{style=mystyle}

\lstset{language=Matlab,backgroundcolor=\color{mygrey}}
\usepackage{lastpage}
\usepackage{fancyhdr}
\pagestyle{fancy}
%%%\lhead{\large{Nik Benko, Peter Creveling}} 
%\chead{\large{\textbf{Automatic Segmentation of Micro CT Images \\  \normalsize{ME EN 6035 Final Project} \\ \small{Nik Benko, Peter Creveling}}}}
%\rhead{\today}
\cfoot{[\thepage\ of \pageref{LastPage}]}
\fancyheadoffset{.5cm}
\setlength{\parindent}{0cm}
\usepackage[left=.5in, right=0.50in, top=1.00in,bottom=1.00in]{geometry}
\usepackage{microtype} 
\usepackage{setspace}
\doublespace
%%%%%%%%%%%%%%%%%%%%%%%%%%%%%%%%%%%%%%%%%%%%%%%%%%%%%%%%%%%%%%%%%%%%%%%%%%
% git testing ii

\begin{document}
\title{ Automatic Segmentation of Micro CT Images  \\ \normalsize{ME EN 6035 Final Project}}
\author{Nik Benko and Peter Creveling}
\maketitle


\begin{abstract} 
In this study, the accuracy of a novel automatic segmentation method was statistically compared to a manual segmentation method. Segmentation was performed on fiber-reinforced ceramic matrix composites in order to extract porosity from the surrounding microscturcture. Four different automatic segmentation methods were investigated. Statistical analysis was performed to determine how the methods compared. It was found that using a combination of pre-segmentation filtering, 3D machine learning segmentation, and post-process edge erosion had the best agreement with manual segmentation of the methods evaluated.
\end{abstract}

\section{Introduction and Background}

Fiber-reinforced ceramic matrix composites (CMCs) are widely used in the aerospace industry due to their high specific strength, high stiffness, and their ability to withstand prolonged exposure to extreme high-temperature and oxidative environments. Despite their increased use, the physics governing their manufacturing, stochastic microstructure, and long-term structural performance remains an active area of research. Understanding the complete life-cycle of CMCs is critical towards improving  existing and future applications of CMCs. In recent years, the emerging practice of in situ X-ray micro-computed tomography (X-ray $\mu$CT) experiments has proven useful for providing a holistic understanding of the behavior of CMCs on multiple length scales \cite{Larson,Bale,Bale2,Cox,Haboub,Marshall}.\\
Unique to X-ray $\mu$CT is the ability to take through-thickness images of specimens in 3-D. This provides much more information compared to other 2-D surface based techniques. However, processing these images to extract information comes at high computational costs. One of the most common imaging processing operations is segmentation, in which pixels are categorized into two more groups that correspond to some characteristic of the material being imaged. For example, CMC images are often segmented to distinguish between fibers and the surrounding epoxy matrix. There exist active areas of research in improving best practices for the segmentation of X-ray $\mu$CT reconstructions. Segmentation techniques vary widely depending on image properties such as noise characteristics, contrast between segments, and edge sharpness. X-ray $\mu$CT of large specimens results in image with significant noise and edge softness, making automatic segmentation challenging. Researchers have largely relied on manual segmentation, in which a user classifies each pixel by hand. While accurate, this method is very time consuming and impractical for processing of large data sets. For this reason the development of an automated method of segmentation is desired.\\
In this study, several automatic segmentation strategies were constructed and tested against a manually segmented data set to determine accuracy. Slices of a 3-D X-ray $\mu$CT image data set were segmented to calculate the porosity of each slice. A statistical validation was performed to compare the differences in segmentation ability between the automatic methods and the manual segmentation method.

\section{Materials and Methods}

\subsection{Composite CMC and X-ray $\mu$CT}
This study is an extension of current research being performed in the Utah Composites lab where X-ray $\mu$CT is being utilized to image the entire multi-step manufacturing process of CMCs combined with \textit{in situ} testing during extreme environments. The end goal is to fully quantify the changes of the CMC microstructure throughout their entire life-cycle via methods of segmentation. The composite system used consisted of a 8HS satin weave CG Nicalon-fiber fabric infiltrated with SMP-10 resin. Imaging was performed using a Zeiss Versa-520 X-ray CT 3D microscope at the Air Force Research Laboratory (Dayton, OH). In this study, images were analyzed from the pyrolysis step of the manufacturing process after which the resin was cured at 1000$^{\circ}$C.

\subsection{Data}
Image were collected with a scan time of 37.3 hours. A small region of interest (ROI) from the entire CMC specimen was selcted for analysis. There were 1980 8-bit images collected with a spatial resolution of 1.21 $\mu$m/voxel. In total, the entire image stack consisted of 22.1GB of information. For this study, a small sub-volume of 500$\times$500$\times$500 pixels analyzed. Within this sub-volume, both manual and automatic segmentation were performed in order to extract porosity (e.g. cracks or voids) from the surrounding microstructure. It is assumed that the manual segmentation based method of porosity is considered ground truth to compare the accuracy of the automatic segmentation.

\subsection{Statistical Analysis} 
Several different statistical techniques were used to evaluate segmentation algorithms. First a one-way ANOVA was conducted on the porosity measurements for each technique to determine whether or not automatic calculations of porosity were significantly different from manual calculations. A multiple comparisons analysis (Tukey's HSD) was then conducted to determine which methods, if any, were significantly different.\\

An additional data set was constructed by calculating the percent error between each of the automatically segmented results and the manually segmented results. This data was no longer dependent on the underlying image and could be treated as a random variable. Normality of data was checked by plotting the histograms for each automatic method. Finally, the four automatic methods segmentations were broken into two categories based on algorithm features with "high" and "low" treatments for each category. A 2 by 2 factorial and 2-way ANOVA were performed to determine which algorithm features (Analysis Dimensions and Edge Erosion), most effected the results. 

\section{Results and Discussion} 
\begin{figure}[H]
	\centering
	\includegraphics[width=0.5\textwidth]{DataHistograms.png}
	\caption{Histogram of porosity distribution for each segmentation method}
	\label{fig:DataHist}
\end{figure}

\begin{figure}[H]
	\centering
	\includegraphics[width=0.5\textwidth]{DifferenceHistograms.png}
	\caption{Histogram of percent error for each of the automatic segmentation methods}
	\label{fig:DiffHist}
\end{figure}

\begin{figure}[H]
	\centering
	\includegraphics[width=0.5\textwidth]{ResponseSurface.png}
	\caption{Response surface model for two different algorithm features}
	\label{fig:Surf}
\end{figure}
\subsection{Error and Uncertainties} 


\section{Conclusion and Future Directions} 

%\begin{figure}[H]
%	\centering
%	\includegraphics[width=1\textwidth]{TestSetUp.png}
%	\caption{Experimental setup of the Split Hopkinson Pressure Bar}
%	\label{fig:TestSetup}
%\end{figure}


\bibliographystyle{ieeetr}
\bibliography{DoE_References}

\end{document}