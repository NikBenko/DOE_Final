
\documentclass[12pt]{article}
\usepackage{amsmath}
\usepackage{amsfonts}
\usepackage{mathrsfs}
\usepackage{lscape}
\usepackage{listings}
\usepackage{graphicx} % Allows for importing of figures
\usepackage{color} % Allows for fonts to be colored
\usepackage{comment} % Allows for comments to be made
\usepackage{accents} % Allows for accents to be made above and below text
%\usepackage{undertilde} % Allows for under tildes to take place for vectors and tensors
\usepackage[table]{xcolor}
\usepackage{array,ragged2e}
\usepackage{hyperref}
\usepackage{framed} % Allows boxes to encase equations and such
\usepackage{subcaption} % Allows for figures to be side-by-side
\usepackage{float} % Allows for images to not float in the document
\usepackage{booktabs}
%\usepackage[margin=0.75in]{geometry}
\usepackage[final]{pdfpages}
\usepackage{enumitem}
\usepackage[section]{placeins}

%%%%%%%%%%%%%%%%%%%%%%%%%  Function used to generate vectors and tensors %%%%%%%%%
\usepackage{stackengine}
\stackMath
\newcommand\tensor[2][1]{%
	\def\useanchorwidth{T}%
	\ifnum#1>1%
	\stackunder[0pt]{\tensor[\numexpr#1-1\relax]{#2}}{\scriptscriptstyle \sim}%
	\else%
	\stackunder[1pt]{#2}{\scriptscriptstyle \sim}%
	\fi%
}
%%%%%%%%%%%%%%%%%%%

\definecolor{mygrey}{rgb}{0.97,0.98,0.99}
\definecolor{codeblue}{rgb}{.2,0,1}
\definecolor{codered}{rgb}{1,0,0}
\definecolor{codegreen}{rgb}{0.3,0.33,0.12}
\definecolor{codegray}{rgb}{0.5,0.5,0.5}
\definecolor{codepurple}{rgb}{0.55,0.0,0.55}
\definecolor{codecyan}{rgb}{0.0,.4,.4}

\lstdefinestyle{mystyle}{
	backgroundcolor=\color{mygrey},   
	commentstyle=\color{codegreen},
	keywordstyle=\color{codeblue},
	stringstyle=\color{codepurple},
	numberstyle=\tiny\color{codegray},
	basicstyle=\footnotesize,
	breakatwhitespace=false,         
	breaklines=true,                 
	captionpos=b,                    
	keepspaces=true, 
	numbers=left,                    
	numbersep=5pt,                  
	showspaces=false,                
	showstringspaces=false,
	showtabs=false,                  
	tabsize=2
}
\lstset{style=mystyle}

\lstset{language=Matlab,backgroundcolor=\color{mygrey}}
\usepackage{lastpage}
\usepackage{fancyhdr}
\pagestyle{fancy}
%%%\lhead{\large{Nik Benko, Peter Creveling}} 
%\chead{\large{\textbf{Automatic Segmentation of Micro CT Images \\  \normalsize{ME EN 6035 Final Project} \\ \small{Nik Benko, Peter Creveling}}}}
%\rhead{\today}
\cfoot{[\thepage\ of \pageref{LastPage}]}
\fancyheadoffset{.5cm}
\setlength{\parindent}{0cm}
\usepackage[left=.5in, right=0.50in, top=1.00in,bottom=1.00in]{geometry}
\usepackage{microtype} 
\usepackage{setspace}
\doublespace
%%%%%%%%%%%%%%%%%%%%%%%%%%%%%%%%%%%%%%%%%%%%%%%%%%%%%%%%%%%%%%%%%%%%%%%%%%
% git testing ii

\begin{document}
\title{ Automatic Segmentation of Micro CT Images  \\ \normalsize{ME EN 6035 Final Project}}
\author{Nik Benko and Peter Creveling}
\maketitle


\begin{abstract} 

\end{abstract}

\section{Introduction and Background}

Fiber-reinforced ceramic matrix composites (CMCs) are widely used in the aerospace industry due to their high specific strength and stiffness material properties, and their ability to withstand prolonged exposure to extreme high-temperature and oxidative environments. Despite their increased use, the fundamental physics governing their manufacturing, stochastic microstructure, and long-term structural performance remains largely undiscovered. Understanding the complete life-cycle of CMCs is critical towards improving currently existing and future applications of CMCs. In recent years, the emerging practice of in situ X-ray micro-computed tomography (X-ray µCT) experiments has proven promising towards providing a holistic understanding into the behavior of CMCs on multiple length scales [1-6].
Unique to X-ray µCT is the ability to image specimens through-thickness in 3-D compared to other surface specific imaging techniques. However, this imaging method comes at a high computational post processing price to analyze terabytes of potential image data. There exist active areas of research in improving best practices for the segmentation of X-ray µCT reconstructions to extract useful properties. Segmentation is defined as the classification of pixels or groups of pixels into multiple segments. These segments often correspond to a physical representation. There is no one segmentation method that works universally. Segmentation is often tailored towards the users’ needs my means of manual or automatic practices. To date, manual segmentation is heavily used due to the knowledge humans use when classifying individual pixels compared to computer based methods. Once trained, computer based methods significantly out produce humans in computational time, and are therefore more desirable.
In this study, a novel segmentation pipeline was developed to accurately segment porosity automatically from the microstructure of CMCs imaged via X-ray µCT. A statistical validation was performed to compare the differences in segmentation ability of porosity between the novel automatic method and the manual segmentation based method. It was hypothesized that the mean percent error in porosity segmentation for the novel segmentation method was equal to zero under the assumption that the manual based method was considered ground truth. The alternative was the mean percent error in automatic porosity segmentation was greater than zero.


\section{Materials and Methods}

\subsection{Composite CMC Micro CT } 

\subsection{Data}

\subsection{Statistical Analysis} 


\section{Results and Discussion} 

\subsection{Error and Uncertainties} 

\section{Conclusion and Future Directions} 

\section{Figures}

%\begin{figure}[H]
%	\centering
%	\includegraphics[width=1\textwidth]{TestSetUp.png}
%	\caption{Experimental setup of the Split Hopkinson Pressure Bar}
%	\label{fig:TestSetup}
%\end{figure}




%\newpage
%\bibliographystyle{IEEEtran}
%\bibliography{Lab3Bib}
\end{document}